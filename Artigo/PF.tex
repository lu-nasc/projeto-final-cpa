%% ------------- Portuguese version ------------ 
\documentclass[portuguese]{sbrt} 
\usepackage[portuguese]{babel} 
\usepackage[utf8]{inputenc} 
\newtheorem{theorem}{Theorem} 
\usepackage[T1]{fontenc} 
\usepackage[pdftex]{hyperref} 
\usepackage{graphicx,url} 
\usepackage[hang]{subfigure} 
\usepackage{psfrag} 
\usepackage{comment} 
\usepackage{graphicx}
\usepackage{amsmath}
\graphicspath{ {./images/} }
%% --------------------------------------------- 
  
%% If writing in English, remove the lines above 
%% and uncomment the lines below 
  
%% ------------- English version --------------- 
%\documentclass[english]{sbrt} 
%\usepackage[english]{babel} 
%\usepackage[utf8]{inputenc} 
%\newtheorem{theorem}{Theorem} 
%% --------------------------------------------- 
  
\begin{document} 
  
\title{Projeto Final GCC253 - Auto Calibração de Câmeras em Visão Estéreo} 
  
\author{Gabriel Martins Silva, Lucas Hideki Sekiya Nascimento, Valter Granato Neto, Vanessa Luana Soares Corrêa. 
\thanks{\centering \textbf{\textit{Complexidade e Pojeto de Algorítmos} -- \today \\ Prof. Douglas H. S. Abreu}}% 
} 
  
\maketitle 

  
\markboth{Projeto Final GCC253 - Complexidade e Projeto de Algorítmos - Departamento de Ciência da Computação (DCC) - Instituto de Ciências Exatas e Tecnológicas (ICET)- UFLA - 2022/1}{} 
  
  
%% If writing in English, remove both 'resumo' and 'chave' 
%% ------------- Portuguese version ------------ 
  
\begin{resumo} 
%\textit{Reduzir, pois o resumo deve conter no máximo 100 palavras.}  
O presente trabalho tem como objetivo estudar alguns métodos de auto-calibração para câmeras utilizadas em visão estéreo. A visão estéreo trabalha com a reconstrução da informação tridimensional de ambientes a partir de imagens capturadas por câmeras com campo de visão sobreposto. Uma vez que esse processo se baseia em captura de imagens, é necessário que as câmeras estejam devidamente calibradas para garantir a confiabilidade e precisão desse método
\end{resumo} 
\begin{chave} 
algoritmos, complexidade, visão computacional, câmeras digitais, revisão bibliográfica, calibração. 
\end{chave} 
  

%% --------------------------------------------- 
  
  

\section{Introdução} 
\label{sec:introducao} 
  
Dentro da computação, a visão estéreo é a área que estuda a reconstrução de dados tridimensionais a partir de imagens bidimensionais \cite{MadalenaMenotti:2020}. Esse tipo de visão está presente em diversos animais - em geral predadores - e baseia-se na comparação de duas imagens parecidas, porém com um pequeno deslocamento: cada olho capta uma imagem levemente diferente do outro. Essas imagens são processadas pelo cérebro e o animal, a partir das imagens bi-dimencionais captadas pelos olhos, consegue atribuir a noção de profundidade ao ambiente que está sendo visualizado.

\newline
Assim como nesses animais, a visão estéreo na computação passa pelas etapas de coleta de imagens e processamento: duas ou mais câmeras captam imagens de um mesmo ambiente, porém de posições diferentes. As imagens são analisadas e a partir delas pode-se recuperar informações tridimensionais daquele ambiente.
 
\newline
Essa concepção também é aplicada em projetos de realidade virtual para causar a ilusão de profundidade e tridimensionalidade, como podemos ver na figura 1
\begin{wrapfigure}
    \begin{center}
        \centering
	    \includegraphics[scale=0.8]{vr}
	    \caption{Figura 1: Implementação de visão estéreo em óculos de realidade virtual}        
    \end{center}
\end{wrapfigure}

\newline
Tendo como base duas câmeras das quais se sabe a posição e o direcionamento, é possível calcular a posição de qualquer ponto dentro do campo de visão dessas câmeras, desde que esse ponto esteja em uma área sobreposta dentro dos campos de visão. Para calcular a posição do ponto é utilizada triangulação, é possível  verificar na figura 2

\begin{wrapfigure}
    \begin{center}
        \centering
	    \includegraphics[scale=1.0]{triangulação}
	    \caption{Figura 2: triangulação de um ponto a partir de imagens sobrepostas na visão estéreo}        
    \end{center}
\end{wrapfigure}

\newline
A visão estéreo é largamente utilizada em sistemas de monitoramento, uma vez que possui uma alta precisão e seu custo de implementação é relativamente baixo. Entretanto, faz-se necessário um processo de calibração para garantir a precisão e qualidade desses sistemas. A auto-calibração é realizada utilizando capturas de cenas estáticas, se que se altere os parâmetros intrínsecos da câmera, dispensando o uso de objetos conhecidos previamente.

\newline
Este artigo se encontra organizado da seguinte forma: na seção II são descritos os parâmetros intrínsecos e extrínsecos das câmeras ~\ref{sec:parametros}. Na seção III são expostos os métodos de calibração propriamente ditos ~\ref{sec:metodos}. Na seção IV discute-se o desempenho dos métodos descritos na seção III ~\ref{sec:desempenho}. Na última seção são retomados os principais pontos abordados ao longo do artigo~\ref{sec:conclusao}.
 

\section{Parâmetros intrínsecos e extrínsecos}
\label{sec:parametros}
Os parametros necessários para realizar os calculos de calibração das camêras foram separados em dois grupos nomeados como parametros intrínsecos e extrínsecos. Sendo intrínsecos sendo referentes a paramêtros internos da camêra fotográfica, e extínsecos externos a camêra e especificamente aos dados da foto digital.

\newline
Lista de parametros intrínsecos:
\begin{itemize}
    \item • Distância focal: A distância focal em pixels.
    \item • Ponto principal: coordenadas do ponto principal.
    \item • Coeficiente de inclinação: define o ângulo entre os eixos x e y do pixel.
    \item • Distorções: coeficientes de distorção de imagem (radial e tangencial).
\end{itemize}

\newline
Lista de parametros extrínsecos:
\begin{itemize}
    \item • Rotação: Um conjunto de matrizes de rotação 3x3
    \item • Translação: Um conjunto de vetores 3x1.
\end{itemize}

Fazendo parte da configuração e preparação para a execução dos algoritmos de calibração, é considerada um \textit{grid} com conjunto de coordenadas \{\mathrm{X,Y,Z}\} e uma matriz \textit{A} que representa a matrix intrínseca da câmera.
  $$
  A =
  \begin{bmatrix}
   \alpha & \gamma & u_{0}\\ 
   0      & \beta  & v_{0}\\
   0      & 0      & 1    
  \end{bmatrix}
  $$
\section{Métodos de calibração}
\label{sec:metodos}

A maioria dos metodos de calibração existentes são baseados em algoritmos de
calibração manual off-line para alvos ou padrões de referencia especificos
o foco do nosso estudo está relacionado aos métodos tradicionais de calibração
O método de calibração tradicional pode calibrar com precisão os parâmetros da câmera
utilizando alvos planos ou estéreos. 

Métodos de modelo planar
Esse metódo tem sido amplamente estudado por Zhang propôs calibrar a câmera com um tabuleiro de xadrex.
A câmera é necessária para visualizar o padrão quadriculado exibido em várias direções
na qual deve identificar seus cantos em uma imagem e calcular toda a divisão do tabuleiro. 
Com isso se tiver qualquer valor errado ou canto não detectado, insere no parâmetro de calibração o erro encontrado e uma nova calibração é feita.

Métodos de calibração baseados em objetos 3D


 






\section{Desempenho} 
\label{sec:desempenho}

\section{Conclusão}
\label{sec:conclusao}
\bibliographystyle{plain} 
\bibliography{ref}

\end{document} 
